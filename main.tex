%% Este trabalho é uma adequação das normas de TCC
%% da Universidade Federal de Mato Grosso (Faculdade de Engenharia) 
%% de acordo com a Norma ABNT.

%% ========================================================================
%% Opções: \documentclass[tipo/curso]{faeng}
%% ------------------------------------------------------------------------
%% Tipo:
%% 	tcc:         Formata documento para TCC
%%	qualitcc:    Formata documento para qualificação de TCC
%% ------------------------------------------------------------------------
%% Curso:
%%   eq:     Engenharia Química
%%   em:     Engenharia de Minas
%%   ec:     Engenharia de Computação
%%   et:     Engenharia de Transportes
%%   eca:    Engenharia de Controle e Automação
%% ------------------------------------------------------------------------
\documentclass[tcc/ec]{faeng}
%% ========================================================================

%% ========================================================================
%% PACOTES
%% ------------------------------------------------------------------------
%% Pacotes fundamentais 
%% ------------------------------------------------------------------------
\usepackage{cmap}			% Mapear caracteres especiais no PDF
\usepackage{lmodern}		% Usa a fonte Latin Modern			
\usepackage{makeidx}        % Cria o indice
\usepackage{hyperref}  		% Controla a formação do índice
\usepackage{lastpage}		% Usado pela Ficha catalográfica
\usepackage{indentfirst}	% Indenta o primeiro parágrafo de cada seção.
\usepackage{nomencl} 		% Lista de simbolos
\usepackage{graphicx}
\usepackage{algorithm}
\usepackage{algorithmic}% Inclusão de gráficos
\usepackage[brazil]{babel}  % Codificação e uso de caracteres especiais.
\usepackage[utf8]{inputenc}

\usepackage{listings}
\lstset{
    language=C++,           % Define a linguagem do código (neste caso, C++)
    basicstyle=\ttfamily,   % Estilo básico da fonte
    numbers=left,           % Coloca números de linha à esquerda
    numberstyle=\tiny,      % Estilo dos números de linha
    breaklines=true,        % Quebra de linha automática
    frame=single,           % Moldura em torno do código
    tabsize=4               % Tamanho da tabulação
}

%% ------------------------------------------------------------------------
%% Pacotes adicionais, usados apenas no âmbito do Modelo faeng
%% ------------------------------------------------------------------------
\usepackage{lipsum}				       % para geração de dummy text
\usepackage[printonlyused]{acronym}
\usepackage{xcolor}
\usepackage{booktabs}
\usepackage{multirow}
%% ========================================================================

%% ========================================================================
%% Informações de dados para CAPA e FOLHA DE ROSTO
%% ------------------------------------------------------------------------
%% Título:
%%	1. Título em português
%%	2. Título em inglês
\titulo{Desempenho de algoritmos clássicos em diferentes linguagens de programação}{Desempenho de algoritmos clássicos em diferentes linguagens de programação}
%% ------------------------------------------------------------------------
%% Autor:
%%	1. Nome completo do autor
%%	2. Formato de nome para bibliografia
\autor{Rafael Simões Martins da Silva Oliveira}{Simões Martins da Silva Oliveira, Rafael}
%% ------------------------------------------------------------------------
%% Cidade
\local{Cuiabá}
%% ------------------------------------------------------------------------
%% Ano de defesa
\data{\today}
%% ------------------------------------------------------------------------
%% Nome do orientador
\orientador{nome do orientador}
%% Nome do coorientador
%\coorientador{Nome completo do coorientador}
%% ========================================================================

%% ========================================================================
%% compila o indice
%% ------------------------------------------------------------------------
\makeindex
%% ========================================================================

%% ========================================================================
%% Compila a lista de abreviaturas e siglas
%% ------------------------------------------------------------------------
\makenomenclature
%% ========================================================================

%% ========================================================================
%% Inserir ficha catalográfica
%%
%% Caso o comando \inserirfichacatalografica seja definido, a ficha catalográfica
%% será inserida atrás da folha de rosto. Caso contrário a página será deixada em branco.
%%
%% CUIDADO: Esta opção deve ser preenchida antes do comando \maketitle
%% ------------------------------------------------------------------------
%\inserirfichacatalografica{fichaCatalografica.pdf}
%% ========================================================================

%% ========================================================================
%% Inserir folha de aprovação
%%
%% Caso o comando \inserirfolhaaprovacao seja definido, a a folha de aprovação
%% será inserida.
%% CUIDADO: Esta opção deve ser preenchida antes do comando \maketitle
%% ------------------------------------------------------------------------
%\inserirfolhaaprovacao{folhaAprovacao.pdf}
%% ========================================================================

%% ========================================================================
%% INÍCIO DO DOCUMENTO
%% ------------------------------------------------------------------------
\begin{document}
%% ------------------------------------------------------------------------
%% ELEMENTOS PRÉ-TEXTUAIS
%% ------------------------------------------------------------------------
\pretextual
%% ------------------------------------------------------------------------
%% Insere Capa, Folha de rosto, Ficha catalográfica (se inserida)
%% e folha de aprovação (se inserida).
%% ------------------------------------------------------------------------
\maketitle
%% ------------------------------------------------------------------------
%% Dedicatória
%% ------------------------------------------------------------------------
\imprimirdedicatoria{Dedicação desse trabalho ao Presidente da República Luiz Inácio Lula da Silva}
%% ------------------------------------------------------------------------
%% Agradecimentos
%% ------------------------------------------------------------------------
\imprimiragradecimentos{
Agradecimentos.
}
%% ------------------------------------------------------------------------
%% Epígrafe
%% ------------------------------------------------------------------------
\imprimirepigrafe{
		``Frase.''\\
		(Autor)
}
%% ========================================================================

%% ========================================================================
%% RESUMO e ABSTRACT
%% ------------------------------------------------------------------------
%% Resumo em português
%% ------------------------------------------------------------------------ 
\begin{resumo}{Palavra-chave. Palavra-chave. Palavra-chave. Palavra-chave. Palavra-chave. Palavra-chave}

 	 Resumo.

\end{resumo}
%% ------------------------------------------------------------------------
%% Resumo em inglês
%% ------------------------------------------------------------------------
\begin{abstract}{Keyword. Keyword. Keyword. Keyword. Keyword. Keyword}

Abstract.
	
\end{abstract}
%% ========================================================================

%% ========================================================================
%% inserir lista de ilustrações
%% ------------------------------------------------------------------------
\listailustracoes
%% ========================================================================

%% ========================================================================
%% inserir lista de tabelas
%% ------------------------------------------------------------------------
\listatabelas
%% ========================================================================

%% ========================================================================
%% inserir lista de abreviaturas e siglas
%% ------------------------------------------------------------------------
\listasiglas{abreviaturas}
%% ========================================================================

%% ========================================================================
%% inserir o sumario
%% ------------------------------------------------------------------------
\sumario
%% ========================================================================

%% ========================================================================
%% ELEMENTOS TEXTUAIS
%% ------------------------------------------------------------------------
\mainmatter
%% ========================================================================

%% ========================================================================
%% Capitulos 
%% ------------------------------------------------------------------------
%% Capítulo externo
\chapter[Introdução]{Introdução}



A semantica 
%% ------------------------------------------------------------------------
%% Capítulo
\chapter[Definição de  Linguagem de Programação]{Capítulo}
Uma Linguagem de programação pode ser entendida como uma forma padrão de comunicação entre ser humano e um computador, no qual a interação se dá por meio desta, de modo semelhante a linguagem  natural, que permite aos seres humanos o estabelecimento de comunicação entre si.

No início, as linguagens de programação eram aplicadas apenas para resolução de problemas específicos de matemática, por vezes complexos para se fazer á mão, mas com o uso de computadores e as linguagens de programação para seu desenvolvimento.

Uma linguagem de programação é um conjunto de de informações estruturadas, com um a semântica e uma sintaxe para implementar um código fonte, que pode ser compilado ou interpretado, e transformado em um programa de computador.

Alan Turing afirma que os computadores foram projetados para decodificação da linguagem humana, considerando o código como uma única linguagem executável

\

\section{Definição de Máquina de Turing}
Uma máquina de Turing é um modelo teórico de um dispositivo inventado pelo matemático e cientista da computação Alan Turing em 1936. Ela foi projetada para entender e formalizar o conceito de computabilidade e desempenhou um papel crucial no desenvolvimento da teoria da computação.

Uma máquina de Turing é composta por uma fita infinita dividida em células, onde cada célula pode conter um símbolo de um alfabeto finito. A máquina de Turing possui um cabeçote que pode ler e escrever símbolos na fita, além de se mover para a esquerda ou direita. A máquina tem um conjunto de estados e uma tabela de regras de transição que determinam como ela se comporta com base no símbolo atual lido, no estado atual e nas regras especificadas.

O modelo de máquina de Turing é uma abstração poderosa que permitiu aos cientistas da computação e matemáticos estudar a natureza da computação, compreender os limites da resolvabilidade algorítmica e desenvolver conceitos fundamentais da ciência da computação, como a noção de algoritmo, computabilidade e decidibilidade. Ela é frequentemente usada para definir formalmente o que é "computável" e para analisar a complexidade computacional de algoritmos e problemas.

\section{Tipo de Linguagem de Programação}

\subsection{Linguagens de programação de alto nível}

\subsection{Linguagens de programação de baixo nível}

\subsection{Linguagens de programação orientadas a Objetos}

\subsection{Linguagens Scripiting}

\subsection{Linguagens Funcionais}

\subsection{Linguagens de Consulta}

\subsection{Linguagens de domínio específico}

\subsection{Linguagurens de programação web}

\section{Paradigmas de Programação}

\subsection{Programação Imperativa}

\subsection{Programação Orientada a Objetos}

\subsection{Programação Lógica}

\subsection{Programação Estruturada}

\subsection{Programação Orientada a Aspectos}

\subsection{Programação Orientada a Eventos}

%% Capítulo
\chapter[Algoritmos clássicos de computação]{Algoritmos Clássicos}

\section{Algoritmos Clássicos de ordenação}

\subsection{Bubble Sort}
O algoritmo Bubble Sort é o método de ordenação mais simples, basicamente faz a comparação entre as adjacências de \textit{array}, ou vetor, com as seguinte lógica, compara o elemento na posição "i" com o elemento da na posição "j", tal que $ i = j+1$.

O Algoritmo em questão possui dois laços de repetição, o laço interno, que é sobre a variável $j$ faz iteração com a comparação dos elementos da posição $i$ com os demais elementos do \textit{array}, em seguida no laço mais externo quando o valor da variável $i$ incrementa, o processo do laço interno é novamente avaliado, com início na posição $j = i+1$ e limite na posição $j = n - 1$, sendo $n$ o comprimento do vetor

Basicamente temos a seguinte estrutura do algoritmo:

\begin{algorithm}
\caption{Bubble Sort}
\begin{algorithmic}
\REQUIRE $(V,n)$
\ENSURE $(V,n) \quad ||  \quad V[1] \leq V[2] \leq \ldots \leq V[n-1] \leq V[n]$
\STATE $i \gets 0$
\STATE $j \gets 0$
\STATE $aux \gets 0$

\FOR{$i \gets 0$ to $n$}
    \FOR{$j \gets 0$ to $n-1$}  
        \IF{$V[j] > V[j+1]$}
            \STATE $aux \gets V[j]$
            \STATE $V[j] \gets V[j+1]$
            \STATE $V[j+1] \gets aux$
        \ENDIF
    \ENDFOR
\ENDFOR 
\end{algorithmic}
\end{algorithm}

Neste ponto temos os seguintes cenários para avaliação:
\begin{enumerate}
    \item Ordenação com números aleatórios no \textit{array}
    \item Ordenação com números em ordem crescente no  \textit{array}
    \item Ordenação com número em ordem decrescente no \textit{array}
\end{enumerate}

\subsection{Análise dos cenário em linguagem compilada,C++}

Em C++, a implementação fica:

\begin{lstlisting}
#include<cstdlib>
#include<ctime>
#include <iostream>


void bubble(int* v, int n) {
	int i, j, aux;
	for (i = 0; i < n; i++) {
		for (j = 0; j < n - 1;j++) {
			if (v[j] > v[j + 1]) {
				aux = v[j];
				v[j] = v[j + 1];
				v[j + 1] = aux;
			}
		}
	}
}

int main()
{
	int first_clock = clock();
	int first_time = time(NULL);
	srand(time(NULL));
	
	int n = 0;
	std::cin >> n;

	printf("%d \n", n);
	int *v = new int[n];

	for (int i = 0; i < n; i++) {
		v[i] = 0 + rand() % n;
	}


	bubble(v, n);



	delete [] v;

	int second_clock = clock();
	int second_time = time(NULL);

	std::cout << "Tempo decorrido (clock): " << (second_clock - first_clock) / (double)CLOCKS_PER_SEC << " segundos" << std::endl;
	std::cout << "Tempo decorrido (time): " << (second_time - first_time) << " segundos" << std::endl;

	return 0;
		 
}
\end{lstlisting}

\subsubsection{Ordenação com números aleatórios no \textit{array}}

%% ------------------------------------------------------------------------
%% Conclusão
\chapter[Conclusão]{Conclusão}
A conclusão também é um capítulo.
%% ========================================================================

%% ========================================================================
%% ELEMENTOS PÓS-TEXTUAIS
%% ------------------------------------------------------------------------
\postextual
%% ========================================================================

%% ========================================================================
%% Referências bibliográficas
%% ------------------------------------------------------------------------

%% ========================================================================

%% ========================================================================
%% Glossário
%% ------------------------------------------------------------------------
%\glossary
%% ========================================================================

%% ========================================================================
%% Apêndices
%% ------------------------------------------------------------------------
%% Inicia os apêndices
%% ------------------------------------------------------------------------
\begin{apendicesenv}
%% Imprime uma página indicando o início dos apêndices





\end{apendicesenv}
%% ========================================================================

%% ========================================================================
%% Anexos
%% ------------------------------------------------------------------------
%% Inicia os anexos
%% ------------------------------------------------------------------------
\begin{anexosenv}
\end{anexosenv}
%% ========================================================================

\end{document}